This Perl script uses the \href{https://ui.adsabs.harvard.edu/}{NASA
astronomy data service (ADS)} citation keys and the
\href{https://github.com/yymao/adstex}{Python \texttt{adstex} package}
so that you can forget about your bibliography management! For citation
commands in your document that use ADS citation keys, i.e., commands
like this \texttt{\textbackslash{}cite\{1958ZA.....46..108B\}}, the
script automatically fetches and caches the bibliography entries from
the ADS database in a managed bibliography file (normally with extension
\texttt{.adskeys.bib}).

Citations with keys that are not present in the ADS database can be
added by providing one or more additional bibliography files, as is done
in the source file for this document:
\texttt{\textbackslash{}bibliography\{manbib.adskeys,custom\}}.

The script is intended for integration with
\href{https://ctan.org/pkg/latexmk/}{Latexmk}, which automates the build
process for LaTeX documents.

\subsection{How it works}\label{how-it-works}

The Perl script
\href{./managed-bibliography.pl}{\texttt{managed-bibliography.pl}} reads
bibliography information from the \texttt{.aux} file which is created by
the LaTeX compiler during a build. The script builds a small
\texttt{.keys.tex} file which contains all the citations in the
\texttt{.tex} files that are part of the build. This file is passed to
the Python \texttt{adstex} package, which updates the managed
bibliography file when new ADS citation keys are found.

Keys that are no longer used in the document are not automatically
removed from the managed bibliography file, but they can be purged by
deleting the managed bibliography file and rebuilding the document. You
can also use the \texttt{delete\_on\_full\_clean} option to have the
managed bibliography file deleted when running \texttt{latexmk\ -C}.
(Note that depending on other settings, Latexmk may keep or delete the
\texttt{.bbl} file during a clean operation.)

An example of how to integrate the script with Latexmk is provided in
the \texttt{latexmkrc} file included in this repository.

\subsection{Test case}\label{test-case}

To build \href{./manbib.tex}{\texttt{manbib.tex}} with Latexmk run the
following command in the terminal:

\begin{verbatim}
latexmk -pdf -bibtex manbib.tex
\end{verbatim}

This will create the file \texttt{manbib.pdf} along with the managed
bibliography file \texttt{manbib.adskeys.bib}. You can clean up the
generated files by running:

\begin{verbatim}
latexmk -C manbib.tex
\end{verbatim}
