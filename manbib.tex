\documentclass{article}
\usepackage{xcolor}
\usepackage{listings} 
\usepackage{hyperref}
\hypersetup{
    colorlinks=true,
    linkcolor=blue,       
    filecolor=blue,      
    urlcolor=blue 
}
\usepackage{aas_macros}

\begin{document}

\section{Managed bibliography with ADS keys} 
This Perl script uses NASA astronomy data service (ADS) citation keys and the \texttt{adstex} Python package so that you can forget about your bibliography management! For citation commands in your document that use ADS citation keys, i.e., commands like this 
\texttt{\textbackslash{}cite\{1958ZA.....46..108B\}}, the script automatically fetches and caches the bibliography entries from the ADS database in a managed bibliography file (normally called \texttt{adstex.keys.bib}).

Citations with keys that are not present in the ADS database can be added by providing one or more additional bibliography files, as is done in the source file for this document: \texttt{\textbackslash{}bibliography\{adstex.keys,custom\}}. 

The script is intended for integration with \texttt{latexmk}, which automates the build process for LaTeX documents.

\subsection*{How it works}
The Perl script \texttt{managed-bibliography.pl} reads bibliography information from the \texttt{.aux} file which is created by the LaTeX compiler during a build. The script builds a small \texttt{keys.tex} file which contains all the citations in the \texttt{.tex} files that are part of the build. This file is passed to the Python \texttt{adstex} package, which updates the managed bibliography file when new ADS citation keys are found. 

Keys that are no longer used in the document are not automatically removed from the managed bibliography file, but they can be purged by deleting the \texttt{adstex.keys.bib} file and rebuilding the document. An example of how to integrate the script with \texttt{latexmk} is provided in the \texttt{latexmkrc} file included in this repository.


\section{Example \texttt{latexmkrc} file}
Here is an example of a \texttt{latexmkrc} file that loads the \texttt{managed-bibliography.pl} script and sets some options:

\lstset{
    language=Perl,
    commentstyle=\color{gray},
    basicstyle=\ttfamily\small,
}
\lstset{
  rangeprefix=\#\ ,       % everything before the label
  rangesuffix=,          % nothing special after
}
\lstinputlisting[
  linerange={Settings-End\ of\ settings}
]{latexmkrc}
% \lstinputlisting[lastline=47]{latexmkrc}
% \lstinputlisting[rangeprefix=\#<*,rangesuffix=*>,linerange={START-END}]{example.py}

\section{Example citations}
Papers that use the ADS citation keys are automatically included in the managed bibliography file \texttt{adstex.keys.bib}.
Here are some examples of such citations:
\cite{%
    1958ZA.....46..108B,%
    1962AJ.....67..471K,%
    1966AJ.....71...64K%
}. 
Here are a few more
\cite{2002Sci...295...82K,2003PASP..115..763C}. 

Works that are not in the ADS database can be cited as usual by providing a custom bibliography file. Here is an example: \cite{mcgrail2004}.


\bibliographystyle{plain}
\bibliography{adskeys,custom}

\end{document}